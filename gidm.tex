\ifx\pdfoutput\undefined % We're running regular LaTeX (not pdftex)
\documentclass[dvips,12pt,twoside]{book}
\RequirePackage[plainpages=false]{hyperref}
\usepackage{color}
\typeout{Use 'fig4latex -l' then 'make' to format graphics for regular latex}
\else % We're running PdfTeX
\documentclass[pdftex,12pt,twoside]{book}
%\usepackage{thumbpdf}
\RequirePackage[pdftex]{hyperref}
\hypersetup{colorlinks=true, %
  plainpages=false, %
  pdfpagelabels, %
  %hyperindex=true, %
  %backref=true, %
  %bookmarks=true, %
  pdftitle={A Gentle Introduction to Discrete Mathematics, version 1.0}, %
  pdfauthor={Joseph E. Fields}, %
  pdfsubject={Discrete Mathematics, Combinatorics, Graph Theory, Discrete Geometry}, %
  %pdfpagelayout=SinglePage, %
  %pdfpagetransition=Dissolve, %
  pdfstartview=FitH, %
  %pdfstartpage=1, %
  %pdffitwindow=true
}
\pdfcompresslevel=9
\typeout{Use 'fig4latex -p' then 'make' to format graphics for pdflatex.}

\fi

\usepackage[english]{babel} %language selection
\selectlanguage{english}

\usepackage{makeidx}
\usepackage{graphicx}
\usepackage{rotating}
\usepackage{amssymb}
\usepackage{amsmath}
\usepackage{amsthm}
\usepackage{color}

\usepackage{ifthen}

%These booleans allow us to generate 3 different books from the same LaTeX source

\newboolean{InTextBook}
\setboolean{InTextBook}{true}
\newboolean{InWorkBook}
\setboolean{InWorkBook}{false}
\newboolean{InHints}
\setboolean{InHints}{false}


%This command puts different amounts of space depending on whether we are
% in the text, the workbook or the hints & solutions manual. 
\newcommand{\twsvspace}[3]{%
 \ifthenelse{\boolean{InTextBook} }{\vspace{#1}}{%
  \ifthenelse{\boolean{InWorkBook} }{\vspace{#2}}{%
   \ifthenelse{\boolean{InHints} }{\vspace{#3}}{} %
   }%
  }%
 }

\newcommand{\wbvfill}{\ifthenelse{\boolean{InWorkBook}}{\vfill}{}}
\newcommand{\textbookpagebreak}{\ifthenelse{\boolean{InTextBook}}{\newpage}{}}
\newcommand{\workbookpagebreak}{\ifthenelse{\boolean{InWorkBook}}{\newpage}{}}
\newcommand{\hintspagebreak}{\ifthenelse{\boolean{InHints}}{\newpage}{}}

\newcommand{\hint}[1]{\ifthenelse{\boolean{InHints}}{{ \color[rgb]{0,0,1} #1} }{}}

%All my customized LaTeX commands are in this file:

\input{my_latex_definitions.tex}

%Things that effect the size and placement of text on the page

\renewcommand{\baselinestretch}{1.3}
\renewcommand{\arraystretch}{.77}

\addtolength{\textheight}{.25in}
\addtolength{\oddsidemargin}{.75in}
\addtolength{\evensidemargin}{-.75in}

\makeindex

\begin{document}

\frontmatter

\title{A Gentle Introduction to Discrete Mathematics\\ {\small Version 1.0}}
\author{Joe Fields}
\date{Southern Connecticut State University}

\maketitle

\clearpage

\rule{0pt}{0pt}

\vfill

\begin{quote}
    Copyright \copyright{}  2014  Joseph E. Fields.
    Permission is granted to copy, distribute and/or modify this document
    under the terms of the GNU Free Documentation License, Version 1.3
    or any later version published by the Free Software Foundation;
    with no Invariant Sections, no Front-Cover Texts, and no Back-Cover Texts.
    A copy of the license is available at the GIDM website in the section 
    entitled ``GNU Free Documentation License''.
\end{quote}

\vfill

\begin{quote}
The latest version of this book is available (without charge) in portable document format at \newline
\rule{0pt}{0pt} \hspace{1in} \verb+http://www.southernct.edu/~fields/+
\end{quote}

\vfill

\clearpage

\rule{0pt}{0pt}

\vfill

\begin{quote}
{\Large \bf Acknowledgments} 

   This is version 1.0 of {\em A Gentle Introduction to Discrete Mathematics}.
  
\end{quote}

\vfill

\clearpage

\tableofcontents

\listoffigures

\listoftables

\input{pref}
\input{pref}

\mainmatter
\part{Introduction}
\section{Introduction}

\subsection{What is discrete math?}

First let's get a silly point of linguistic confusion out of the way.

\hspace{.4in} \parbox{
{\bf discreet} (comparative more {\bf discreet}, superlative most {\bf discreet})
\begin{enumerate}
    \item Respectful of privacy or secrecy; quiet; diplomatic.
\begin{quote}
        With a discreet gesture, she reminded him to mind his manners.
        John just doesn't understand that laughing at Mary all day is not very discreet.
\end{quote}
    \item Not drawing attention, anger or challenge; inconspicuous.
\end{enumerate}
}

versus

\hspace{.4in} \parbox{
\begin{enumerate}
    \item {\bf discrete} (comparative more {\bf discrete}, superlative most {\bf discrete)}
\begin{enumerate}
    \item Separate; distinct; individual; non-continuous.
    \item That can be perceived individually and not as connected to, or part of something else.
 \end{enumerate}
 
 (plus several other technical meanings)
}

So, a discrete mathematician is not one who knows how to keep a secret, but rather someone
who deals with mathematical entities that are distinct, separate, non-continuous.  The real numbers
in the interval $(0,1)$ lie in the realm of continuous math; the rational numbers in the same interval
are a discrete collection.


\section{Problems}
In this section we ask you to explore a variety of questions that will
serve to introduce some of the fundamental areas within discrete mathematics. 

Thoughts:
Calculate the odds of getting the various poker hands in 5 card stud
Resolve the Monty Hall dilemma
K\"{o}nigsberg and variations
Map coloring
Hamilton's dodecahedron problem
Latin squares and orthogonal Latin squares

\part{Counting}
\include{combi/basic}
\include{combi/recursion}
\include{combi/gen_funcs}
\part{Graphs}
\section{Introduction to Graph Theory}


\include{graphs/coloring}
\include{graphs/pebbling}
\part{Geometry}
\include{geo/finite_fields}
\section{Affine and projective geometries}

\include{geo/eccs}
\include{geo/designs}

\bibliographystyle{plain}
\bibliography{main}
\addcontentsline{toc}{chapter}{References}
\phantomsection
\markboth{REFERENCES}{REFERENCES}%

\addcontentsline{toc}{chapter}{Index} 
\phantomsection
\markboth{INDEX}{INDEX}
\printindex

\end{document}

