\section{Introduction}

\subsection{What is discrete math?}

First let's get a silly point of linguistic confusion out of the way.

\hspace{.4in} \parbox{
{\bf discreet} (comparative more {\bf discreet}, superlative most {\bf discreet})
\begin{enumerate}
    \item Respectful of privacy or secrecy; quiet; diplomatic.
\begin{quote}
        With a discreet gesture, she reminded him to mind his manners.
        John just doesn't understand that laughing at Mary all day is not very discreet.
\end{quote}
    \item Not drawing attention, anger or challenge; inconspicuous.
\end{enumerate}
}

versus

\hspace{.4in} \parbox{
\begin{enumerate}
    \item {\bf discrete} (comparative more {\bf discrete}, superlative most {\bf discrete)}
\begin{enumerate}
    \item Separate; distinct; individual; non-continuous.
    \item That can be perceived individually and not as connected to, or part of something else.
 \end{enumerate}
 
 (plus several other technical meanings)
}

So, a discrete mathematician is not one who knows how to keep a secret, but rather someone
who deals with mathematical entities that are distinct, separate, non-continuous.  The real numbers
in the interval $(0,1)$ lie in the realm of continuous math; the rational numbers in the same interval
are a discrete collection.

